\documentclass[15pt,a4paper]{article}
\usepackage[portuguese]{babel}
\usepackage[utf8]{inputenc}
\usepackage{indentfirst}
\usepackage{graphicx}
\usepackage{verbatim}
\begin{document}
\setlength{\textwidth}{16cm}
\setlength{\textheight}{22cm}

\title{\Huge\textbf{Implementação em Prolog do Jogo Oware}\linebreak\linebreak\linebreak
\Large\textbf{Relatório Intercalar}\linebreak\linebreak
\includegraphics[height=6cm, width=7cm]{feup.pdf}\linebreak \linebreak
\Large{Mestrado Integrado em Engenharia Informática e Computação} \linebreak \linebreak
\Large{Programação em Lógica}\linebreak
}

\author{\textbf{Grupo 35:}\\ André Freitas - ei10036 \\ Rui Gonçalves - ei10100 \\\linebreak\linebreak \\
 \\ Faculdade de Engenharia da Universidade do Porto \\ Rua Roberto Frias, s\/n, 4200-465 Porto, Portugal \linebreak\linebreak\linebreak
\linebreak\linebreak\vspace{1cm}}
%\date{Junho de 2007}
\maketitle
\thispagestyle{empty}

%************************************************************************************************
%************************************************************************************************

\newpage

\section*{Resumo}
Este relatório tem o objetivo de descrever a fase de inicial de análise da implementação do Jogo Oware numa linguagem de programação lógica em matemática que é o Prolog. O jogo em questão é de tabuleiro, jogando-se com sementes, sendo muito popular na República do Gana. Apesar da simplicidade das regras tem um forte componente estratégico.

%************************************************************************************************
%************************************************************************************************

%*************************************************************************************************
%************************************************************************************************

\section{Introdução}
Pretende-se explorar as capacidades do Prolog para representar um jogo e as suas regras através deste trabalho. Um jogo é uma excelente maneira de enriquecer o conhecimento na representação e estruturação dos dados bem como a aplicação de regras reais traduzidas nesta linguagem. \\
\indent A implementação será em modo de texto, não sendo muito refinada no sentido da apresentação mas sim na componente de funcionalidades. \\
\indent Neste documento pretende-se apresentar a descrição do problema, a representação dos estados do jogo em estruturas conhecidas em Prolog, a representação das jogadas, a visualização do tabuleiro e os apectos de desenvolvimento do projeto. \\\\
Devem ser incluídas referências bibliográficas correctas e completas (consultar os docentes em caso de dúvida). Páginas da wikipedia não são consideradas referências válidas \cite{CodigoSite, CodigoLivro}.

Todas as figuras devem ser referidas no texto. %\ref{fig:codigoFigura}


%Código para inserção de figuras
%\begin{figure}[h!]
%\begin{center}
%escolher entre uma das seguintes três linhas:
%\includegraphics[height=20cm,width=15cm]{path relativo da imagem}
%\includegraphics[scale=0.5]{path relativo da imagem}
%\includegraphics{path relativo da imagem}
%\caption{legenda da figura}
%\label{fig:codigoFigura}
%\end{center}
%\end{figure}

Ajuda:

\textit{Para escrever em itálico}

\textbf{Para escrever em negrito}

Para escrever em letra normal

``Para escrever texto entre aspas''

Para fazer parágrafo, deixar uma linha em branco.

Como fazer bullet points:

\begin{itemize}
\item Item1
\item Item2
\end{itemize}

Como enumerar itens:

\begin{enumerate}
\item Item 1
\end{enumerate}

\begin{quote}``Isto é uma citação''\end{quote}

\section{Descrição do Problema}
O jogo é portanto o Oware, que tem origem africana sendo mais jogado na República do Gana como aquele que é o jogo nacional extremamente popular. 


\begin{figure}[africanPeople!]
\begin{center}
\includegraphics[scale=0.5]{awale.jpg}
\caption{Pessoas jogando tradicionalmente o Oware}
\label{fig:peoplePlaying}
\end{center}
\end{figure}

Descrever sucintamente o jogo, a sua história e, principalmente, as suas regras. Devem ser criadas/utilizadas imagens apropriadas para explicar o funcionamento do jogo.

\section{Representação do Estado do Jogo}
Descrever a forma de representação do estado do tabuleiro (tipicamente uma lista de listas), com exemplificação em Prolog de posições iniciais do jogo, posições intermédias e finais.

\section{Representação de um Movimento}
Descrever a forma de representação dos diversos tipos de jogadas (movimentos) permitidos no jogo. Só é necessário apresentar os cabeçalhos dos predicados que serão utilizados para as diferentes jogadas (que ainda não precisam de estar implementados).

\section{Visualização do Tabuleiro}
Descrever a forma de visualização do tabuleiro em modo de texto e os predicados Prolog construídos para o efeito. O código (predicado) desenvolvido, deve receber como parâmetro a representação do tabuleiro (estado do jogo) e permitir visualizar, no ecrã, em modo de texto, o estado do jogo. Deve ser incluída no relatório, pelo menos, uma imagem demonstrando a visualização em modo de texto do tabuleiro.

\section{Conclusões e Perspectivas de Desenvolvimento}
Que conclui da análise do jogo e da pesquisa bibliográfica realizada? Como vai ser desenvolvido o trabalho? Que parte (\%) do trabalho estima que falta fazer?

\clearpage
\addcontentsline{toc}{section}{Bibliografia}
\renewcommand\refname{Bibliografia}
\bibliographystyle{plain}
\bibliography{myrefs}

\newpage
\appendix
\section{Nome do Anexo A}
Código Prolog implementado (representação do estado, cabeçalhos dos predicados de jogada e predicado que permite a visualização simples, em modo de texto, do tabuleiro).

\end{document}

\documentclass[15pt,a4paper]{article}
\usepackage[portuguese]{babel}
\usepackage[utf8]{inputenc}
\usepackage{indentfirst}
\usepackage{graphicx}
\usepackage{verbatim}
\usepackage{float}
\usepackage{caption}
\usepackage{subcaption}
\begin{document}
\setlength{\textwidth}{16cm}
\setlength{\textheight}{22cm}

\title{\Huge\textbf{Implementação em Prolog do Jogo Oware}\linebreak\linebreak\linebreak
\Large\textbf{Relatório Final}\linebreak\linebreak
\includegraphics[height=6cm, width=7cm]{feup.pdf}\linebreak \linebreak
\Large{Mestrado Integrado em Engenharia Informática e Computação} \linebreak \linebreak
\Large{Programação em Lógica}\linebreak
}

\author{\textbf{Grupo 35:}\\ André Freitas - ei10036 \\ Rui Gonçalves - ei10100 \\\linebreak\linebreak \\
 \\ Faculdade de Engenharia da Universidade do Porto \\ Rua Roberto Frias, s\/n, 4200-465 Porto, Portugal \linebreak\linebreak\linebreak
\linebreak\linebreak\vspace{1cm}}
%\date{Junho de 2007}
\maketitle
\thispagestyle{empty}
\newpage
\section*{Resumo}
Este relatório tem o objetivo de descrever a implementação do Jogo Oware numa linguagem de programação lógica em matemática que é o Prolog. O jogo em questão é de tabuleiro, jogando-se com sementes, sendo muito popular na República do Gana. Apesar da simplicidade das regras tem um forte componente estratégico.\\
O jogo foi implementado com recurso a listas para representar o tabuleiro e para representar as propriedades dos jogadores, estando organizando em módulos de manipulação do tabuleiro, rotinas de jogo, inteligência artificial, interfaces e predicados de teste. Existem 2 níveis de bots que jogam, respectivamente, aleatoriamente e com base numa heurística.

%************************************************************************************************
%************************************************************************************************

%*************************************************************************************************
%************************************************************************************************

\section{Introdução}
Pretende-se explorar as capacidades do Prolog para representar um jogo e as suas regras através deste trabalho. Um jogo é uma excelente maneira de enriquecer o conhecimento na representação e estruturação dos dados bem como a aplicação de regras do jogo traduzidas nesta linguagem. \\
\indent A implementação em questão é em modo de texto, não sendo muito refinada no sentido da apresentação mas sim na componente das funcionalidades. \\
\indent Neste documento pretende-se apresentar a descrição do problema, a representação dos estados do jogo em estruturas conhecidas em Prolog, a representação das jogadas, a visualização do tabuleiro, a inteligência artificial e os aspetos de desenvolvimento do projeto.

\section{Descrição do Problema}
O Oware é dos jogos de tabuleiro mais antigos do mundo, tendo sido inventado há mais de 7 mil anos. É jogado por todo o Globo e não existem certezas relativamente à sua origem, porém, atribui-se a sua autoria tradicionalmente ao continente Africano. Atualmente é o jogo mais popular na República do Gana sendo um fenómeno nacional.

\begin{figure}[h!]
\begin{center}
\includegraphics[scale=20]{awale.jpg}
\caption{Pessoas jogando tradicionalmente o Oware}
\label{fig:traditional}
\end{center}
\end{figure}

Como se pode constatar pela Figura 1, existe um tabuleiro com 2x6 cavidades onde se colocam sementes ou feijões. Existem muitas interpretações das regras deste jogo, pelo que iremos adotar apenas a que é mais conhecida. Assim, o jogo começa com 4 sementes em cada buraco. Os jogadores jogam alternadamente, e em cada jogada tira-se as sementes de um buraco da nossa linha de jogo e vai-se distribuindo as sementes no sentido anti-horário.
\begin{figure}[H]
\begin{center}
\includegraphics[scale=0.5]{oware.jpg}
\caption{Distribuição das sementes}
\label{fig:peoplePlaying}
\end{center}
\end{figure}
 \indent Quando distribuirmos a última semente e se colocarmos num buraco do adversário e esse sítio ficar com 2 ou 3 sementes, capturamos essas sementes. O jogo termina quando um jogador capturar 25 sementes ou ambos os dois jogadores capturarem 24 sementes (empate).

\subsection{Ilustração de jogadas}
Nas seguintes ilustrações serão descritas duas situações de jogos fundamentais. Pede-se especial atenção para as ilustrações, dado que é necessário estar atento às situações que descrevem para perceber o jogo.\\
\indent Nesta primeira situação o jogo está a começar e o jogador vai distribuir a segunda cavidade de sementes. Assim, as sementes ficaram distribuídas e não houve captura. De notar novamente que o sentido da distribuição das sementes é anti-horário e que o jogador só pode retirar as sementes da sua linha.
\begin{figure} [H]
        \centering
        \begin{subfigure}[f]{0.3\textwidth}
                \centering
                \includegraphics[scale=0.2]{iniciodoJogo.png}
				\caption{Estado Inicial do Jogo}
                \label{fig:inicioJogo}
        \end{subfigure}%
        \quad  \quad
        \begin{subfigure}[f]{0.3\textwidth}
                \centering
                \includegraphics[scale=0.2]{semCaptura.png}
				\caption{Jogada sem Captura}
				\label{fig:semCaptura}
        \end{subfigure}
\end{figure}

A outra jogada importante é a da captura de sementes, ou seja, quando um jogador ao distribuir a última semente calha numa linha do seu adversário, ficam 2 ou 3 sementes nesse sítio, capturando-as. Atenção que se a meio da distribuição das sementes conseguir este número nas cavidades, não as pode capturar, por isso é só na semente final. Destaca-se esta situação dado que existem várias interpretações relativas a esta regra.

\begin{figure} [H]
        \centering
        \begin{subfigure}[f]{0.3\textwidth}
                \centering
                \includegraphics[scale=0.2]{antesCaptura.png}
				\caption{Antes da Captura}
                \label{fig:inicioJogo}
        \end{subfigure}%
        \quad  \quad
        \begin{subfigure}[f]{0.3\textwidth}
                \centering
                \includegraphics[scale=0.2]{captura.png}
				\caption{Após a Captura de Sementes}
				\label{fig:semCaptura}
        \end{subfigure}
\end{figure}

Pode-se constatar assim a simplicidade do jogo, mas apesar disso requer um componente estratégico para que se possa ganhar, tentando prever-se uma panóplia de jogadas possíveis para vencer o adversário e maximizar as nossas capturas.

\section{Arquitectura do Sistema}
Ora, como referido anteriormente, o sistema é decomposto em módulos o que permite um melhor isolamento no desenvolvimento e manutenção futura do código. Assim sendo, os módulos são os seguintes:
\begin{enumerate}
  \item oware.pl - possui todos os predicados inerentes à rotina de jogo;
  \item owareBoard.pl - contém os predicados para manipular e aceder à estrutura do tabuleiro;
  \item owareCLI.pl - alberga os predicados que interagem na consola de texto com o utilizador;
  \item owareAI.pl - módulo que contém os predicados para os bots;
  \item owareTest.pl - módulo que possui testes considerados pertinentes;
\end{enumerate}
Destaca-se a atenção na análise do código dos predicados gameRoutine, aiPlay e playSeeds, que são o ponto de partida para a compreensão da implementação do jogo.
\subsection*{Visualizador 3D}
Para a implementação com o visualizador 3D, será criado um módulo que terá o nome "owareServer.pl", usando a comunicação por socket do Sicstus com o seguinte protocolo:
\begin{verbatim}
     +----------------+                            +----------------+
     |   LAIG APP     |   beginGame human bot2     |  Oware Server  |
     |----------------|+-------------------------->|----------------|
     |                |                      ack   |                |
     |                |<--------------------------+|                |
     |                |                            |                |
     |                |                            |                |
     |                |              playerTurn 1  |                |
     |                |<--------------------------+|                |
     |                |  gameStatus (board) 1 2    |                |
     |      C++       |<--------------------------+|     Prolog     |
     |                |  playerChooses 1           |                |
     |                |+-------------------------->|                |
     |                |                            |                |
     |                |                            |                |
     |                |                            |                |
     |                |      endGame victory 1     |                |
     |                |<--------------------------+|                |
     +----------------+                            +----------------+
\end{verbatim}
Assim sendo, a linguagem das mensagens, é a seguinte:
\begin{enumerate}
\item beginGame (player1Type) (player2Type) - indica a informação para iniciar o jogo;
\item ack - confirma o início do jogo;
\item playerTurn (1 or 2) - indica qual é o jogador que vai jogar;
\item gameStatus (board) (player1 score) (player2 score) - indica o estado do jogo;
\item playerChoose (1-6) - indica qual a posicao a jogar;
\item endGame (victory (1 or 2)) or (draw) - sinaliza o fim do jogo.
\end{enumerate}

Dado que as mensagens que são passadas entre os sistema são relativamente simples, este formato é suficiente, porém, se se tratasse de um sistema que necessitasse de algo mais complexo, faria todo o sentido usar um formato para representar estruturas de dados como JSON, pois do lado da aplicação de LAIG em C++ existem bibliotecas para tal.


\section{Interface com o Utilizador}
Para a visualização do tabuleiro são usados três predicados: um principal para representar a totalidade do tabuleiro, um para apresentar o logótipo do Jogo e outro para imprimir uma linha de sementes.

 \begin{verbatim}
printBoard([H|[Th|_]],P1Score,P2Score)
printLogo
printBoardLine([H|T])
\end{verbatim}

\begin{figure}[h!]
	\begin{center}
	\includegraphics[scale=1]{printBoardExample.png}
	\caption{Começo do jogo}
	\label{fig:Comeco}
	\end{center}
\end{figure}

Estes predicados implicam sempre uma lista de listas de 6 elementos para que o tabuleiro apareça de forma correcta.
\newpage
\section{Conclusões e Perspetivas de Desenvolvimento}
\indent A implementação do Oware em termos de estrutura de dados é simples, porém, 
requer maior esforço na implementação dos predicados que envolvem a distribuição das peças, principalmente o do cálculo da sequência de posições. Existem diversas interpretações das regras e outros nomes para o Oware, por isso tentamos adotar a versão que é mais conhecida, que já foi aliás implementada em telemóveis.\\

\indent Relativamente ao processo de desenvolvimento de trabalho, iremos adotar um desenvolvimento iterativo com uma metodologia ágil que é o Scrum. Assim sendo, temos o seguinte Backlog com a estimação dos pontos de esforço, estando ordenado pela prioridade dos itens:

\begin{enumerate}
  \item Representar o tabuleiro numa estrutura [N/A]
  \item Apresentar o tabuleiro em modo de texto [2 pts]
  \item Criar funções básicas para listas [2 pts]
  \item Deve ser possível adicionar e remover sementes na estrutura do tabuleiro [3 pts]
  \item Implementar o cálculo da sequência aquando de uma distribuição de sementes [3 pts]
  \item Implementar um predicado para jogar as sementes [5 pts]
  \item Deve existir um predicado que avalia se há uma captura, devolvendo a posição das sementes [5 pts]
  \item Implementar um predicado para ler o input do teclado [2 pts]
  \item Implementar a rotina de jogo [8 pts]
  \item Deve existir um bot que consegue determinar que posição das sementes deve jogar a partir do reconhecimento da linha do adversário [8 pts]
\end{enumerate}

Ora assumindo este Backlog e tendo em conta que já implementamos até ao quarto item falta 80\% do esforço deste trabalho, usando uma estimativa mais razoável. Tento em conta isto, não se traduz em horas de trabalho mas sim esforço e ultrapassar dificuldades. Os pontos mais críticos serão a rotina de jogo e a implementação da inteligência artificial do Bot.
\newpage
\section{Bibliografia}
\begin{thebibliography}{9}

\bibitem{oware}
  The Oware Society.
  2010.
  \emph{ Oware- Played all over the World}.
  Acedido a 4 de Outubro de 2012.
  http://www.oware.org.
  
  \bibitem{Figura 1}
  Awale.jpg.
  2006.
  \emph{A game of awale}.
  Acedido a 4 de Outubro de 2012.
  http://upload.wikimedia.org/wikipedia/commons/1/14/Awale.jpg.
  
   \bibitem{Figura 2}
	oware.jpg.
  \emph{Playing Oware in Ghana}.
  Acedido a 4 de Outubro de 2012.
  http://exploringafrica.matrix.msu.edu/teachers/events/oware.jpg.
  
   \bibitem{Figura 3}
	Easy Oware
	2012.
  \emph{Play the classic strategy game from Africa}.
  Acedido a 2 de Outubro de 2012.
  http://itunes.apple.com/br/app/easy-oware/id408219960?mt=8.

\end{thebibliography}

\newpage
\appendix
\section{Código Implementado}
 \begin{verbatim}
\end{verbatim}
\end{document}
